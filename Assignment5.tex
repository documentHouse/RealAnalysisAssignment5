%\documentclass[11pt,reqno]{amsart}
\documentclass[11pt,reqno]{article}
\usepackage[margin=.8in, paperwidth=8.5in, paperheight=11in]{geometry}
%\usepackage{geometry}                % See geometry.pdf to learn the layout options. There are lots.
%\geometry{letterpaper}                   % ... or a4paper or a5paper or ... 
%\geometry{landscape}                % Activate for for rotated page geometry
%\usepackage[parfill]{parskip}    % Activate to begin paragraphs with an empty line rather than an indent7
\usepackage{graphicx}
\usepackage{pstricks}
\usepackage{amssymb}
\usepackage{epstopdf}
\usepackage{amsmath}
\usepackage{subfigure}
\usepackage{caption}
\pagestyle{plain}
%\renewcommand{\topfraction}{0.3}
%\renewcommand{\bottomfraction}{0.8}
%\renewcommand{\textfraction}{0.07}
\DeclareGraphicsRule{.tif}{png}{.png}{`convert #1 `dirname #1`/`basename #1 .tif`.png}

\title{Real Analysis $\mathbb{I}$: \\ Assignment 5}
\author{Andrew Rickert}
\date{Started: April 20, 2011 \\ \hspace{1pt} Ended: April ??  2011}                                           % Activate to display a given date or no date

\begin{document}
\maketitle


% Page 1
\begin{flushleft} 
\textbf{Class 18.100B} - Problem 1\\
\rule{500pt}{1pt}\\
\end{flushleft} 

First we show that $X$ being closed implies completeness. Since a Cauchy sequence in $X$ is a Cauchy sequence in $M$ where $X$ is a subset and $M$ is complete we know that the sequence converges. Since the sequence consists of points of $X$ then every neighborhood of the point of convergence contains elements of $X$, that is it is a limit point. Because $X$ is closed it contains the limit point so every Cauchy sequence in $X$ converges to a point in $X$ and so the set is complete.\\
\indent For the inverse we need to show that if $X$ is complete then $X$ is closed. Consider a limit point $x \in X$. By the definition of a limit point there is $p_1 \in X$ such that $d(p_1,x) < \frac{1}{2}$. Similarly we may find a $p_2 \in X$ such that $d(p_2,x) < \frac{1}{4}$ and so on. In this way we produce a sequence in $X$ such that $d(p_n,x) < \frac{1}{2^n}$. This sequence converges to a point $p$. Since every convergent sequence is a Cauchy sequence by a theorem in Rudin and $X$ is complete the point of convergence must be in $X$ so $p \in X$ and $X$ is closed.

\vspace{15pt}
\begin{flushleft} 
\textbf{Class 18.100B} - Problem 2\\
\rule{500pt}{1pt}\\
\end{flushleft} 

\noindent Part a) We need to show that if $(x_{2n})$ and $(x_{2n-1})$ both converge to the same limit then $(x_n)$ converges and vice versa. Concluding that $(x_{2n})$ and $(x_{2n-1})$ converge when $(x_n)$ follows from the fact that every subsequence of convergent series converges to the same limit. The other conclusion that $(x_n)$ converges if $(x_{2n})$ and $(x_{2n-1})$ do goes as follows.\\
\indent Since $(x_{2n})$ converges there exists an $N_1$ such that for given $\epsilon$ we have $|x_{2n}-x'| < \epsilon$ for $n > N_1$ where $x'$ is the limit. Similarly for $(x_{2n-1})$ we have $N_2$ such that  $|x_{2n-1}-x'| < \epsilon$ for $n > N_2$. If we take $N = \text{max} (2N_1,2N_2)$ then for $n > N$ we have $|x_n -x'| < \epsilon$ since either $x_n = x_{2n'}$ or $x_n = x_{2n'-1}$ for some $n'$. This shows that $(x_n)$ converges by virtue of the convergence of $(x_{2n})$ and $(x_{2n-1})$.\\

\noindent Part b) Now we need to show that $(x_n)$ converges if $(x_{2n})$,$(x_{2n-1})$, and $(x_{5n})$ all converge. We suppose that $\lim_{n \to \infty}(x_{2n}) = \alpha$ and, $\lim_{n \to \infty}(x_{2n-1}) = \beta$ and $\alpha \neq \beta$. \\
\indent There is a subsequence of $(x_{5n})$ that is a subsequence of $(x_{2n})$ since if $n = 2k$ we have $5n = 5(2k) = 2(5k) = 2j$ where $j = 2k$. So for even $n$, then subsequence of $(x_{5n})$ is a subsequence of $(x_{2n})$. Since every subsequence of a convergent sequence converges to the same limit then we have $\lim_{n \to \infty}x_{5n} = \lim_{k \to \infty} x_{5(2k)} = \lim_{k \to \infty}x_{2(2k)} = \lim_{n \to \infty}x_{2n} = \alpha$.\\
\indent Similarly there is a subsequence of $(x_{5n})$ that is a subsequence of $(x_{2n-1})$   since if $n = 2k-1$ then $5n = 5(2k-1) = 10k -5 = 2(5k-2)-1= 2j-1$ where $j=5k-2$. By the same reasoning as the previous paragraph we get $\lim_{n \to \infty}x_{5n} = \lim_{k \to \infty} x_{5(2k-1)} = \lim_{k \to \infty}x_{2(2k-1)-1} = \lim_{n \to \infty}x_{2n-1} = \beta$. This shows that $\alpha = \beta$ contrary to our hypothesis so  $(x_{2n})$ and $(x_{2n-1})$ must converge to the same limit and by part a this must be the limit of the origin sequence $(x_n)$.

\vspace{15pt}
\begin{flushleft} 
\textbf{Class 18.100B} - Problem 3\\
\rule{500pt}{1pt}\\
\end{flushleft} 

We need to show that  $\limsup (x_n + y_n) \le \limsup x_n + \limsup y_n$. First we point out that the set of all subsequences is closed which means by a theorem of Rudin that the supremum is included in the set. This means there exists a subsequence that limits to this value which is the limit supremum. So, if $(x_{m_k})$ is any subsequence and $(x_{n_k})$ is a sequence that limits to the limit supremum then  $\lim_{k \to \infty}(x_{m_k}) \le \lim_{k \to \infty}(x_{n_k})$ by the definition of the upper limit.\\
\indent if we let $(x_{m_k} + y_{m_k})$ be the limit supremum sequence of ($x_n + y_n$)  and $(x_{n_k})$ and $(y_{n_k})$ be the limit supremum sequences of their individual sequences respectively then we have the following calculation that gives the result
\begin{eqnarray*}
\limsup (x_n + y_n) & = & \lim_{k \to \infty}(x_{m_k} + y_{m_k}) \\
			      & = & \lim_{k \to \infty} x_{m_k} + \lim_{k \to \infty}y_{m_k}\\
                                          & \le & \lim_{k \to \infty} x_{n_k} + \lim_{k \to \infty}y_{n_k}\\
			      & = & \limsup \; x_n + \limsup \;y_n
\end{eqnarray*}

The observation that if $(x_{m_k})$ is any sequence then $\lim_{k \to \infty}(x_{m_k}) \ge \liminf\; x_n$ allows the reasoning in the previous paragraphs to show the result for the $\liminf$.\\
\indent Now we need to show that if $(x_n)$ converges then we get equality instead of the inequality. Since $x_n$ converges by hypothesis every subsequence converges to the same limit. No suppose that $\{n_k\}$ is a set of integers such that the subsequence of $(y_n)$ converges to the $\limsup$. That is, $\lim_{k \to \infty}y_{n_k} = \limsup y_n$. If we use the same set of integers for $(x_n)$ then we have the subsequence $(x_{n_k} + y_{n_k} )$. Since by definition the limit supremum is the least upper bound of subsequential limits we have the following \[\limsup(x_n + y_n) \ge \lim_{k \to \infty}(x_{n_k} + y_{n_k})\]
Since $x_n$ converges and every subsequence converges to the same limit, a limit supremum sequence of this sequence must equal to same limit as any subsequence. So we have 
\begin{eqnarray*}
\limsup(x_n + y_n) & \ge & \lim_{k \to \infty}(x_{n_k} + y_{n_k})\\
                                  & = & \lim_{k \to \infty}x_{n_k} + \lim_{k \to \infty}y_{n_k}\\
                                  & = & \limsup x_n + \lim_{k \to \infty}y_{n_k} \; \text{by the previous comment}\\
                                 & = & \limsup x_n + \limsup y_n \;  \text{by the definition of the subsequence}\;\{n_k \}
\end{eqnarray*}
 Since both $\limsup(x_n + y_n) \ge \limsup x_n + \limsup y_n$ and $\limsup(x_n + y_n) \le \limsup x_n + \limsup y_n$ from the previous part of the problem we have  $\limsup(x_n + y_n) = \limsup x_n + \limsup y_n$. The reasoning for the $\liminf$ is the same with appropriate inequality signs reversed.

\vspace{15pt}
\begin{flushleft} 
\textbf{Class 18.100B} - Problem 4\\
\rule{500pt}{1pt}\\
\end{flushleft} 

Let's quickly show that if $\liminf x_n = \limsup x_n = x'$ then $\lim_{n \to \infty} x_n = x'$. Suppose the hypothesis is given then from a theorem in Rudin if there is an $x > \limsup x_n$ then there is an $N_1$ such that $x_n < x$. For the infimum we have the fact that if $x < \liminf x_n$ then there is an $N_2$ such that $x_n > x$. If we denote $x^* = \liminf x_n = \limsup x_n$ then we let $x = x^* + \epsilon$ for the limit supremum and $x = x^* - \epsilon$ and take $N = \text{max}\;(N_1,N_2)$ and we have $x^* - \epsilon < x_n < x^* + \epsilon$ for $n > N$. This is to say that the sequence converges to the common value for the limit supremum and limit infimum.\\
\indent From looking at the inequality we are intended to prove $\liminf x_n \le \liminf a_n \le \limsup a_n \le \limsup x_n$ we can see by the result just shown that if $x_n \to x$ then $a_n \to x$. This is so because if  $x_n \to x$ then $\liminf x_n = \limsup x_n$ since they are both subsequences of a convergent sequence. Also if $\liminf a_n < \limsup a_n$ then we violate the inequality we are suppose to prove because then we have $x \le \liminf a_n < \limsup a_n \le x$ which is impossible. From our previous remarks $\liminf a_n = \limsup a_n$ shows $a_n \to a$ so we have $x \le a \le x$ which implies that $x = a$ or $a_n \to x$ if $x_n \to x$. So our goal becomes to try to show the inequality and the main result will follow.\\
\indent Because $ \liminf a_n < \limsup a_n $ follows directly from the definitions of the upper and lower limits and greatest/least upper bounds we will focus on showing $\limsup a_n < \limsup x_n$ and then comment on how the result can be adapted to the lower limit.\\
\indent We start as suggested in the problem by letting $x^* = \limsup x_n$ and $K = \{ k \in \mathbb{N} \; | \; x_k \ge x^* + \epsilon \}$. We note that $K$ is finite for otherwise we would have an infinite number of points in the interval $[x^* + \epsilon, \text{sup} \; x_n]$. This compact set will have a limit point and so there is a subsequence whose limit is $\ge x^* + \epsilon$ contrary to the definition of $x^*$. Next we define $\mathcal{S}_n = \{ i \in \mathbb{N} \; | \; i \in K \; \text{and} \; i \le n\}$ and $\mathcal{T}_n = \{ i \in \mathbb{N} \; | \; i \notin K \; \text{and} \; i \le n\}$. Either $x_k \ge x^* + \epsilon$ or $x_k < x^* + \epsilon$ for all $k \in \mathbb{N}$ so $x_n \in \mathcal{S}_n$ or $x_n \in \mathcal{T}_n$. $\mathcal{S}_n$ only includes points in $K$ so we know that it is finite because $K$ is. This means that we can always rearrange the sum $\sum_n x_n$ so that the terms with indices in $\mathcal{S}_n$ will be at the front because a finite rearrangement does not change the value of a series. Since the other terms must be $\mathcal{T}_n$ we have the sum $\sum_n x_n = s_n + t_n$ where $s_n = \sum_{i \in \mathcal{S}_n}x_i$ and $t_n = \sum_{i \in \mathcal{T}_n}x_i$. \\
\indent By the definition of $\mathcal{T}_n$ we know that if $k \in \mathcal{T}_n$ then $k \notin K$ so $x_k < x^* + \epsilon$. So $t_ n = \sum_{k \in \mathcal{T}_n} x_k < n(x^* + \epsilon)$ 
which gives $\frac{t_n}{n} < x^* + \epsilon$. This gives $\limsup \frac{t_n}{n} \le x^* + \epsilon$. Also, because $\mathcal{S}_n$ is finite we know that the sum of all $x_k$ with $k \in K$ is a finite value as well and so is an upper bound $S$ on the sum $s_n$. Therefore we have $\frac{s_n}{n} < \frac{S}{n}$ for all $n$. This gives $\limsup \frac{s_n}{n} = 0$.\\
\indent Using the result from the previous problem we derive:
\begin{equation}
\limsup a_n = \limsup(\frac{s_n}{n} + \frac{t_n}{n}) \le  \limsup \frac{s_n}{n} +\limsup \frac{t_n}{n} = 0 + x^* + \epsilon
\end{equation}

We have that $\limsup a_n \le x^* + \epsilon$ but $\epsilon$ is arbitrary so $\limsup a_n \le x^*$. Otherwise we would have $\limsup a_n - x^* = \beta > 0$ then let $\epsilon = \frac{\beta}{2}$ and we violate the inequality  $\limsup a_n \le x^* + \epsilon$. Since $x^* = \limsup x_n$ we have $\limsup a_n \le \limsup x_n$ as was to be shown. By defining $K' = \{ k \in \mathbb{N} \; | \; x_k \le x' + \epsilon \}$ with $x' = \liminf x_n$ and repeating the previous arguments using $
\liminf$ instead of $\limsup$ we get $\liminf x_n \le \liminf a_n$ which completes the inequality.\\
\indent A counterexample to show that if $a_n$ may converge but $x_n$ will not is the sequence $x_n = (1,0,1,0,1,0,1,\cdots)$. From inspection it is clear that $a_n = (1,\frac{1}{2},\frac{2}{3},\frac{1}{2},\frac{3}{5},\frac{1}{2},\frac{4}{7},\cdots)$ which is to say $a_n = \frac{1}{2}$ for even indices and $a_n = \frac{n}{2n-1}$ for odd indices. From this we see $a_n \to \frac{1}{2}$ and that $x_n$ does not converge.

\vspace{15pt}
\begin{flushleft} 
\textbf{Class 18.100B} - Problem 5\\
\rule{500pt}{1pt}\\
\end{flushleft} 

Our first step is to show that the sequence $x_{n+1} = 1 - \sqrt{1-x_n}$ is bounded. We are given that $1 > x_1 > 0$. Now assume that $1 > x_n > 0$. Then we have
\begin{eqnarray*}
 0 < 1 - x_n < 1 &\implies& 1 > \sqrt{1- x_n} > 0 \\
                          &\implies& -1 < -\sqrt{1-x_n} < 0 \\
                          &\implies& 0 < 1 - \sqrt{1-x_n} < 1 \\
                          &\implies& 0 < x_{n+1} < 1
\end{eqnarray*}
This completes the induction and shows that $x_n$ is bounded by 0 and 1. We can now show that the sequence decreases. Since $x_n >$ we have the following calculation
\begin{eqnarray*}
x_n > 0 & \implies & 1 > 1 - x_n \\
	 & \implies & 1-x_n > (1-x_n)^2 \quad \text{since $1 > x_n$ implies $1 - x_n > 0$} \\
	 & \implies & x_n > 1 - \sqrt{1-x_n} \\
 	& \implies & x_n > x_{n+1} \\
\end{eqnarray*}
This shows that $(x_n)$ is strictly decreasing. Since the sequence is both decreasing and bounded below we know from a theorem in Rudin that the sequence converges. This allows us to perform the following calculation.
\[
x = \lim_{n \to \infty} x_{n+1} = \lim_{n \to \infty} 1 - \sqrt{1 - x_n} = 1- \sqrt{1 - x}
\]

So $x = 1 - \sqrt{1-x}$ where $x = \lim_{n \to \infty} x_{n+1}$, solving the equation for $x$ yields $x = 0$ as was to be shown.\\
\indent The last part of the problem is to show that $\lim_{n \to \infty}\frac{x_{n+1}}{x_n} \to \frac{1}{2}$. We perform to the following calculation to get an appropriate form for the desired limit.
\begin{eqnarray*}
x_{n+1} &=& 1 - \sqrt{1 - x_n} \\
1 - x_{n+1} &=& \sqrt{1 - x_n} \\
(1 - x_{n+1})^2 &=& 1 - x_n \quad \text{Since $1 - x_n > 0$ implies that $1 - x_n \neq 0$}\\
1 - 2 x_{n+1} + x_{n+1}^2 &=& 1 - x_n \\
2 x_{n+1} - x_{n+1}^2 &=& x_n \\
\frac{x_{n+1}}{x_n}(2 - x_{n+1})&=& 1 \\
\frac{x_{n+1}}{x_n}&=& \frac{1}{2 - x_{n+1}} \\
\end{eqnarray*}
 From this we see that \[\lim_{n \to \infty} \frac{x_{n+1}}{x_n} = \lim_{n \to \infty}\frac{1}{2-x_{n+1}} = \frac{1}{2}\] since as was already established $\lim_{n \to \infty} x_{n+1} = 0$. This completes the problem.
\vspace{15pt}
\begin{flushleft} 
\textbf{Class 18.100B} - Problem 6\\
\rule{500pt}{1pt}\\
\end{flushleft} 


\vspace{15pt}
\begin{flushleft} 
\textbf{Class 18.100B} - Problem 7\\
\rule{500pt}{1pt}\\
\end{flushleft} 


\vspace{15pt}
\begin{flushleft} 
\textbf{Class 18.100B} - Extra Problem \\
\rule{500pt}{1pt}\\
\end{flushleft} 

\end{document}  